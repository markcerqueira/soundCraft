% LOrkCraft Submission for NIME 2013 
%
% Using the LaTeX Template for NIME 2013

\documentclass{nime-document-class}

\begin{document}

\conferenceinfo{NIME'13,}{May 27 -- 30, 2013, KAIST, Daejeon, Korea.}

% use \projectName{} to refer to the project in the paper
% makes changing it later much easier - just change it here!
\newcommand{\projectName}{LOrkCraft}

\title{\projectName{}: Making Music While Playing Starcraft 2}

\numberofauthors{3}
\author{
% 1st. author
\alignauthor
Mark Cerqueira\\
       \affaddr{Smule, Inc.}\\
       \affaddr{577 College Avenue}\\
       \affaddr{Palo Alto, California}\\
       \email{mark@smule.com}
% 2nd. author
\alignauthor
Spencer Salazar\\
       \affaddr{Center for Computer Research in Music and Acoustics (CCRMA)}\\
       \affaddr{Stanford University}\\
       \email{spencer@ccrma.stanford.edu}
% 3rd. author
\alignauthor
Ge Wang\\
       \affaddr{Center for Computer Research in Music and Acoustics (CCRMA)}\\
       \affaddr{Stanford University}\\
       \email{ge.stanford.edu}
}

\maketitle
\begin{abstract}
\projectName{} is a collection of utilities that enable real-time data gathering from a Starcraft 2 match to external audio servers via Open Sound Control allowing the match itself to become a live performance piece. This paper details the technical and aesthetic decisions made during the development of \projectName{}, covering which data is collected during the game and discussing the sonic mappings explored.
\end{abstract}

\keywords{NIME, Starcraft 2, game}

\section{Introduction}
Introduce Starcraft 2 and the premise for taking game data and making music with it.

\section{Background}
Talk about background.

\section{Related Work}
Discuss previous work.

\section{System Overview}
Describe the system as a whole.

\subsection{Creating the Custom Map}
The pain of creating a custom map.

\subsection{Parsing and Sending Game Data}
Ruby script and OSC message format.

\subsection{Sonification of Game Data}
What we did on the audio end.

\subsection{Conclusion}
What we want people to do with \projectName{}.

% Acknowledgments are optional
\section{Acknowledgments}
We would like to thank Tasteless and Artosis for being awesome.

% Place this command where you want to balance the columns on the last page. 
% \balancecolumns 

\end{document}
