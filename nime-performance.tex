% LOrkCraft Performance Submission for NIME 2013 
%
% Using the LaTeX Template for NIME 2013

\documentclass{nime-document-singlecolumn-class}

\begin{document}

\conferenceinfo{NIME'13,}{May 27 -- 30, 2013, KAIST, Daejeon, Korea.}

% use \projectName{} to refer to the project in the paper
% makes changing it later much easier - just change it here!
\newcommand{\projectName}{LOrkCraft}

\title{\projectName{}: Making Music While Playing Starcraft 2 - Performance Proposal}

\numberofauthors{3}
\author{
% 1st. author
\alignauthor
Mark Cerqueira\\
       \affaddr{Smule, Inc.}\\
       \affaddr{577 College Avenue}\\
       \affaddr{Palo Alto, California}\\
       \email{mark@smule.com}
% 2nd. author
\alignauthor
Spencer Salazar\\
       \affaddr{Center for Computer Research in Music and Acoustics (CCRMA)}\\
       \affaddr{Stanford University}\\
       \email{spencer@ccrma.stanford.edu}
% 3rd. author
\alignauthor
Ge Wang\\
       \affaddr{Center for Computer Research in Music and Acoustics (CCRMA)}\\
       \affaddr{Stanford University}\\
       \email{ge@ccrma.stanford.edu}
}

\maketitle


\section{Title and Description}
Title and detailed description of the proposed performance. Include a score and video/audio documentation where possible.

\section{Participants and Submitters}
\noindent {\bf Mark Cerqueira} is a Software Engineer at Smule, a mobile application development company focused on building social music experiences.
He graduated from Princeton University where he researched improvements for networked laptop orchestra performances and created the Laptop Orchestra Network Toolkit.
He has extensive experience developing for the iOS and Android platforms, including applications such as Glee Karaoke, Magic Piano, Magic Guitar, and Magic Fiddle.

{\bf Spencer Salazar} is a doctoral student at Stanford's CCRMA, researching computer-based forms of music performance and experience.
In his past he has created new software and hardware interfaces for the ChucK audio programming language, developed prototype consumer electronics for top technology companies, architectured large-scale social music interactions for Smule, an iPhone application developer, and composed for laptop and mobile phone ensembles.
He has worked on mobile applications ranging from Ocarina, a virtual wind instrument for iPhone, to I Am T-Pain, which gives iPhone users the ability to auto-tune themselves in real-time.
I Am T-Pain is consistently one of the top 10 music apps in the iPhone App Store.

{\bf Ge Wang} is an Assistant Professor at Stanford University in the Center for Computer Research in Music and Acoustics (CCRMA), and researches programming languages and interactive software systems for computer music, mobile and social music, and education at the intersection of computer science and music. Ge is the author of the ChucK audio programming language, the founding director of the Stanford Laptop Orchestra (SLOrk) and of the Stanford Mobile Phone Orchestra (MoPhO). Concurrently, Ge is the Co-founder and Chief Creative of Smule, and the designer of the iPhone's Ocarina and Magic Piano. 

\section{Performers/Instruments/Technologies Used}
Number of performers and the instruments and technologies that will be used.

\section{Requirements for the Venue}
Any requirements for the venue. Diagrams of the preferred stage setup and signal routing are recommended.

\section{Evidence of Feasibility}
Evidence of the feasibility of the performance. Include documentation of past performances or related works that demonstrate the submitter's capabilities to implement the proposed performance.

\section{List of Equipment To Be Provided}
A list of any equipment that needs to be provided by the conference organizers.

\section{Instrumental Performers}
This performance does not require any instrumental performers to be provided by the organizers.

% Place this command where you want to balance the columns on the last page. 
% \balancecolumns 

\end{document}
